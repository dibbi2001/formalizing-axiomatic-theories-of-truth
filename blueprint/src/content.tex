% In this file you should put the actual content of the blueprint.
% It will be used both by the web and the print version.
% It should *not* include the \begin{document}
%
% If you want to split the blueprint content into several files then
% the current file can be a simple sequence of \input. Otherwise It
% can start with a \section or \chapter for instance.

\begin{definition}[$\mathcal{L}_T$]
    \label{def:LT}
    We define the langauge $\mathcal{L}_T$ as the language resulting from adding the predicate symbol $T$ to the language $\mathcal{L}$ of \texttt{PA}. 
\end{definition}

\begin{definition}[\texttt{PAT}]
    \label{def:PAT}
    \uses{def:LT}
    We define the system \texttt{PAT} as the system of Peano arithmetic formulated in $\mathcal{L}_T$ including the induction schema for each formula of the language $\mathcal{L}_T$.
\end{definition}

\begin{definition}[\texttt{TB}]
    \label{def:TB}
    \uses{def:PAT}
    The system \texttt{TB} comprises all axioms of \texttt{PAT}. Moreover all sentences of the form $T\ulcorner\varphi\urcorner$ are axioms of the system where $\varphi$ is a sentence of the language of $\mathcal{L}$.
\end{definition}

\begin{definition}[conservativity]
    \label{def:cons}
    \uses{def:LT}
    A truth theory $...$ [\textcolor{red}{how do you get this symbol?}] in the language $\mathcal{L}_T$ is conservative over a theory $\mathcal{S}$ formulated in language $\mathcal{L}$ without the truth predicate if and only if all $...$-theorems $\varphi$ in the language $\mathcal{L}$ are also theorems of $\mathcal{S}$.
\end{definition}

\begin{theorem}
    \label{thm:tb-cons}
    % \lean{[the right reference]}
    \texttt{TB} is conservative over \texttt{PA}.
\end{theorem}

\begin{proof} \uses{def:cons, def:TB}  We show how to transform any given \texttt{TB}-proof of an arithmetical formula into a \texttt{PA}-proof of the same formula. 

\end{proof}
