% In this file you should put the actual content of the blueprint.
% It will be used both by the web and the print version.
% It should *not* include the \begin{document}
%
% If you want to split the blueprint content into several files then
% the current file can be a simple sequence of \input. Otherwise It
% can start with a \section or \chapter for instance.

Halbach says: ``I will not clearly distinguish between theories and the systems that generate them,'' but perhaps we should. The problem is that for Halbach's purposes only a system's axioms matter, but perhaps we need the system's inference rules as well; I'm not sure. My preference goes out to the word 'system' as we should explicitly discuss proofs when talking about conservativity. Theorems, being ``a set of formulae closed under first-order logical consequence,'' (Halbach, p. 29) do not contain any information what rules govern sentence generation in proofs. Or are we allowed to define a formal proof simply as ``a finite sequence of sentences, each of which is an axiom of $\mathcal{S}$ or from the preceding sentences according to the rules of first-order inference''? I don't know, but something feels imprecise about the phrase 'first-order inference'.

\begin{definition}[formal system]
    A formal system $\mathcal{S}$ in a language $\mathcal{L}$ is a tuple $\mathcal{S} = \langle \mathcal{A}, \mathcal{R} \rangle$, where 
    \begin{itemize}
        \item $\mathcal{A}$ is a set of axioms, i.e. sentences in $\mathcal{L}$ and 
        \item $\mathcal{R}$ is a set of rules (functions?) for generating sentences from other sentences.
    \end{itemize}
\end{definition}

\begin{definition}[formal proof]
    \label{def:formal-proof}
    A formal proof in a logical system $\mathcal{S}$ is a finite sequence of sentences, where each sentence is an axiom of $\mathcal{S}$, an assumption, or follows from the application of one of $\mathcal{S}$'s rules of inference to previous sentences in the sequence (Wikipedia: Formal proof).
\end{definition}

\begin{definition}[provability]
    \label{def:provable-pa}
    \uses{def:formal-proof, def:PA}
    A formula $\varphi$ is provable in a proof system $\mathcal{S}$ if and only if there exists a formal proof $\mathcal{P}$ in $\mathcal{S}$, such that $\mathcal{P}$ contains no assumptions and $\varphi$ is the last sentence of $\mathcal{P}$.
\end{definition}

\begin{definition}[$\mathcal{L}_T$]
    \label{def:LT}
    We define the language $\mathcal{L}_T$ as the language resulting from adding the predicate symbol $T$ to the language $\mathcal{L}$ of \texttt{PA}. 
\end{definition}

\begin{definition}[\texttt{PAT}]
    \label{def:PAT}
    \uses{def:LT}
    We define the system \texttt{PAT} as the system of Peano arithmetic formulated in $\mathcal{L}_T$ including the induction schema for each formula of the language $\mathcal{L}_T$.
\end{definition}

\begin{definition}[\texttt{TB}]
    \label{def:TB}
    \uses{def:PAT}
    The system \texttt{TB} comprises all axioms of \texttt{PAT}. Moreover all sentences of the form $T\ulcorner\varphi\urcorner \leftrightarrow \varphi$ are axioms of the system where $\varphi$ is a sentence of the language of $\mathcal{L}$ and $\ulcorner \varphi \urcorner$ is the numeral of $\varphi$'s Gödel code.
\end{definition}

\begin{definition}[conservativity]
    \label{def:cons}
    \uses{def:LT}
    A truth system $...$ [\textcolor{red}{how do you get this symbol?}] in the language $\mathcal{L}_T$ is conservative over a system $\mathcal{S}$ formulated in language $\mathcal{L}$ without the truth predicate if and only if all sentences $s \in \mathcal{L}$ provable by $...$ are also provable by $\mathcal{S}$.
\end{definition}

\begin{lemma}[finite axioms in TB]
    \label{lem:finit-ax-tb}
    \uses{def:TB}
    In a proof in \texttt{TB} only finitely many axioms can occur.
\end{lemma}

\begin{proof}
    \uses{def:formal-proof, def:TB} 
    Let $p$ be a proof in \texttt{TB}. Then, by Definition \ref{def:formal-proof}, $p$ is a finite sequence of sentences each of which is an axiom of \texttt{TB} or follows from the preceding sentences according to first-order rules of inference. The amount of axioms the proof contains is less or equal to the total number of sentences. Hence, the number of axioms in the proof should also be finite. 
\end{proof}

\begin{theorem}
    \label{thm:tb-cons}
    % \lean{[the right reference]}
    \texttt{TB} is conservative over \texttt{PA}.
\end{theorem}

\begin{proof} \uses{def:cons, def:TB} 
%We show how to transform any given \texttt{TB}-proof of a formula in $\mathcal{L}$ into a \texttt{PA}-proof of the same formula. 
Let $\psi$ be a formula in $\mathcal{L}$ that is provable in TB by a proof $p$. Then, by Lemma \ref{lem:finit-ax-tb}, we have that $p$ contains finitely many sentences as axioms. Let $n$ be the number of disquotation sentences occurring as axioms in the proof. Then, every axiom $p$ is either (i) an axiom of \texttt{PA}, or (ii) an instance of the induction schema, or (iii) for some $i \leq n$, a sentence of the form: 
\begin{align*}
    T\ulcorner \varphi_i \urcorner \leftrightarrow \varphi_i.
\end{align*}
As axioms of type (i) and (ii) are sentences of $\mathcal{L}$, we need only show how to transform any given disquotation axiom into a logically equivalent sentence of $\mathcal{L}$. Now, let $\tau(x)$ be the formula: 
\begin{align*}
    (x = \ulcorner \varphi_1 \urcorner \land \varphi_1) \lor ... \lor (x = \ulcorner \varphi_k \urcorner \land \varphi_k).
\end{align*}
By Lemma \ref{lem:Tphi-eq-tau}, we have that $T\ulcorner \varphi \urcorner \iff \tau(\ulcorner \varphi \urcorner)$. Hence, we can replace all occurences of the truth predicate $T$ in the given \texttt{TB}-proof with $\tau$. Then the disquotation ...
\end{proof}
    