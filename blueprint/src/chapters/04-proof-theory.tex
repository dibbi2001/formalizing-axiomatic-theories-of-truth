\chapter{Proof Theory}\label{chapter:proof-theory}
As our notion of proof we use a sequent calculus which allows for the introduction of formulas from theories as axioms. For the derivation rules and a logical axiom we employ the sequent calculi G3cp + G3c (see \cite{negri:2001} pages 49 and 67 respectively). There are different approaches for developing sequent calculi that can be used with theories (see for example \cite[Chapter 6]{negri:2001}). All these approaches destroy the structural properties of sequent calculi, like the admissibility of cut, but make proofs about derivations from theories as straightforward as possible. As proof theory is not this project's main concern we choose a naive theoretical axiom: one where sequents with a sentence from the relevant theory in its consequent can readily be added as axioms.
\begin{definition}[Sequent Calculus]\label{def:Seq-Calc}
  \lean{Calculus.Derivation}\leanok
  \uses{def:FO-Language,def:FO-Sentence,def:FO-Formula,def:BF-Substitution,def:FO-Neg,def:FO-And,def:FO-Or,def:FO-Ex}
    Let $\mathcal{L}$ be a formal language, and $Th$ some set of $Th \subseteq \mathcal{S}(\mathcal{L})$ functioning as axioms. Then the inference rules of sequent calculus are

\noindent \textbf{Theoretical Axiom:}

if $A \in Th$ then $\Gamma \Rightarrow A, \Delta$

\begin{center}
    \textbf{G3cp}
\end{center}
\noindent \textbf{Logical Axiom:}

$\Gamma, P \Rightarrow P, \Delta$

\noindent \textbf{Logical Rules:}
\begin{multicols}{2}
\[
\frac{\Gamma \Rightarrow A, \Delta \quad \Gamma \Rightarrow B, \Delta}{\Gamma \Rightarrow A \land B, \Delta}\tag{$R\land$}
\]

\[
\frac{\Gamma, A, B \Rightarrow \Delta}{\Gamma, A \land B \Rightarrow \Delta}\tag{$L\land$}
\]

\[
\frac{\Gamma \Rightarrow A, B, \Delta}{\Gamma \Rightarrow A \lor B, \Delta}\tag{$R\lor$}
\]

\[
\frac{\Gamma, A \Rightarrow \Delta \quad \Gamma, B \Rightarrow \Delta}{\Gamma, A \lor B \Rightarrow \Delta}\tag{$L\lor$}
\]

\[
\frac{\Gamma, A \Rightarrow B, \Delta}{\Gamma \Rightarrow A \supset B, \Delta}\tag{$R\supset$}
\]

\[
\frac{\Gamma \Rightarrow A, \Delta \quad \Gamma, B \Rightarrow \Delta}{\Gamma, A \supset B \Rightarrow \Delta}\tag{$L\supset$}
\]

\[
\frac{}{\bot, \Gamma \Rightarrow \Delta}\tag{$L\bot$}
\]

\end{multicols}

\begin{center}
    \textbf{G3c}
\end{center}
\begin{multicols}{2}

\[
\frac{\Gamma \Rightarrow A[y/x], \Delta}{\Gamma \Rightarrow \forall x A, \Delta}\tag{$R\forall$}
\]

\[
\frac{\Gamma, A[t/x], \forall x A \Rightarrow \Delta}{\Gamma, \forall x A \Rightarrow \Delta}\tag{$L\forall$}
\]

\[
\frac{\Gamma, A[y/x] \Rightarrow \Delta}{\Gamma, \exists x A \Rightarrow \Delta}\tag{$L\exists$}
\]

\[
\frac{\Gamma \Rightarrow \exists x A, A[t/x], \Delta}{\Gamma \Rightarrow \exists x A, \Delta}\tag{$R\exists$}
\]

\end{multicols}

\end{definition}

In order to work with the set of all derivations from some theory with introduce the concept of the set of all derivation of some sequent from some theory. Note that in Lean this is immediate from the definition of the type of derivations.

\begin{definition}[$\mathcal{D}$: the Set of All Derivations]\label{def:Derivation}
\lean{Calculus.Derivation}\leanok
\uses{def:Seq-Calc}
Let $Th$ be some theory in some language $\mathcal{L}$, and $\Gamma, \Delta$ finite sets such that $\Gamma, \Delta \subseteq \mathcal{F}(\mathcal{L},\mathbb{N})$. Then we denote the set of all trees that consists only of rules from the Sequent Calculus (Definition \ref{def:Seq-Calc}) with $Th$ as theory and have as root $\Gamma \Rightarrow \Delta$ by $\mathcal{D}(Th,\Gamma,\Delta)$.
\end{definition}

\begin{definition}[$Th \vdash \Gamma \Rightarrow \Delta$: Sequent is Provable from Theory]\label{def:Sequent-Provable}
\lean{Calculus.sequent_provable}\leanok
\uses{def:Derivation,def:FO-Theory}
We call a sequent $\Gamma \Rightarrow \Delta$ provable from some theory $Th$, denoted $Th \vdash \Gamma \Rightarrow \Delta$, if there exists some derivation $d \in \mathcal{D}(Th,\Gamma,\Delta)$.
\end{definition}

\begin{definition}[$Th \vdash \varphi$: Formula Provable from Theory]\label{def:Formula-Provable}
\lean{Calculus.formula_provable}\leanok
\uses{def:Sequent-Provable,def:FO-Theory}
We call a formula $\varphi$ provable from some theory $Th$, denoted $Th \vdash \varphi$, if there exists some derivation $d \in \mathcal{D}(Th,\emptyset,\{\varphi\})$.
\end{definition}

\begin{definition}[MetaRules]\label{def:MetaRules}
\uses{def:Seq-Calc}
We define many meta rules which should be proved for the sequent calculus, including
\begin{multicols}{2}
\[
\frac{\Gamma \Rightarrow \Delta}{\Gamma, A \Rightarrow \Delta} \tag{$L_{wk}$}
\]

\[
\frac{\Gamma \Rightarrow \Delta}{\Gamma \Rightarrow \Delta, A} \tag{$R_{wk}$}
\]

\[
\frac{}{\Gamma \Rightarrow (t_1 = t_1), \Delta} \tag{$R=$}
\]

\[
\frac{\Gamma \Rightarrow A \supset B, \Delta \quad \Gamma \Rightarrow B \supset A, \Delta}{\Gamma \Rightarrow A \leftrightarrow B, \Delta} \tag{$R\leftrightarrow$}
\]

\[
\frac{\Gamma \Rightarrow A \leftrightarrow B, \Delta}{\Gamma \Rightarrow A \supset B, \Delta} \tag{$R \leftrightarrow elim\_to\_right$}
\]

\[
\frac{\Gamma \Rightarrow A \leftrightarrow B, \Delta}{\Gamma \Rightarrow B \supset A, \Delta} \tag{$R \leftrightarrow elim\_to\_left$}
\]
\end{multicols}
\end{definition}



