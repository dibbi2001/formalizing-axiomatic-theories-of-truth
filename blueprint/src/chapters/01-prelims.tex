\chapter{Preliminaries}

\begin{definition}[formal language] \label{def:formal-language}
    Let $\Sigma$ be a set of symbols. Then a formal language $\mathcal{L}$ is a subset of $\Sigma^*$.
\end{definition}

% \begin{definition}[first-order logical vocabulary]\label{def:FO-logical-vocab}
%     The \textit{first-order logical vocabulary}, is a structure $\mathcal{S} = (\mathcal{V},\neg,\mathcal{B},=,\mathcal{Q},\mathcal{P}, \top, \bot)$ where
%     \begin{enumerate}
%         \item $\mathcal{V} = \{x,y,z,...\}$ is the \textit{set of variables},
%         \item $\neg$ is the \textit{negation operator},
%         \item $\mathcal{B} = \{\land,\lor,\to,\leftrightarrow\}$ is the \textit{set of binary sentential operators},
%         \item $=$ is the \textit{identity predicate},
%         \item $\mathcal{Q} = \{\forall, \exists\}$ is the \textit{set of quantifiers},
%         \item $\mathcal{P} = \{(,)\}$ is the \textit{set of parentheses},
%         \item $\top$ is called \textit{top} and
%         \item $\bot$ is called \textit{bottom}.
%     \end{enumerate}
% \end{definition}

\begin{definition}[variables]\label{def:FO-variables}
    We define the set of variables for a first order language as the finite set $\mathcal{V} = \{x,y,z,...\}$.
\end{definition}

\begin{definition}[first-order signature]\label{def:FO-signature}
    A first-order signature is a structure $\mathcal{S} = (\mathcal{C},\mathcal{F},\mathcal{R},ar)$ such that:
    \begin{enumerate}
        \item $\mathcal{C}$ is a set of \textit{constant symbols},
        \item $\mathcal{F}$ is a set of \textit{function symbols},
        \item $\mathcal{R}$ is a set of \textit{predicate symbols} and
        \item $ar : \mathcal{F} \cup \mathcal{R} \to \mathbb{N}$ is a function that maps each function and predicate symbol to a natural number, it \textit{arity}.
    \end{enumerate}
\end{definition}

\begin{definition}[$\mathcal{L}_{PA}$: the language of peano artihmetic]\label{def:LPA}
    \uses{def:FO-language, FO-signature}
    \lean{LPA}
    The language of Peano arithmetic $\mathcal{L}_{PA}$ is the first-order language corresponding to the signature $(\{0\},\{S,+,\cdot\},\{=\})$.
\end{definition} 

\begin{definition}[primitive term]\label{def:primitive-term}
    Let $\mathcal{S} = (\mathcal{C},\mathcal{F},\mathcal{R},ar)$ a first-order signature. Then, we define the corresponding set of primitive terms $\mathcal{P}_{\mathcal{S}} = \mathcal{C} \cup \mathcal{V}$.
\end{definition}

\begin{definition}[first-order term]\label{def:FO-term}
    \uses{def:FO-logical-vocab, def:FO-signature}
    Let $\mathcal{S} = (\mathcal{C},\mathcal{F},\mathcal{R},ar)$ be a first-order signature. Then, the set $\mathcal{T}_{\mathcal{S}}$ of \textit{terms} corresponding to $\mathcal{S}$ is recursively defined as the smallest set $X$ such that:
    \begin{enumerate}
        \item $\mathcal{P}_{\mathcal{S}} \subseteq X$ and
        \item if $t_1,...,t_n \in X$, $f \in \mathcal{F}$ and $ar(f) = n$, then $f(t_1,...,t_n) \in X$.
    \end{enumerate}
\end{definition}

\begin{definition}[first-order language]\label{def:FO-language}
    \uses{def:FO-logical-vocab, def:FO-signature, def:FO-term}
    Let $\mathcal{S} = (\mathcal{C},\mathcal{F},\mathcal{R},ar)$ be a first-order signature. Then the first order language $\mathcal{L}_{\mathcal{S}}$ is recursively defined as the smallest set $X$ such that:
    \begin{enumerate}
        \item \begin{enumerate}
            \item $\top,\bot \in X$,
            \item if $R \in \mathcal{R}$, $ar(R) = n$ and $t_1,...,t_n \in \mathcal{T}_{\mathcal{S}}$, then $R(t_1,...t_n) \in X$,
            \item if $t_1,t_2 \in \mathcal{T}_{\mathcal{S}}$, then $t_1 = t_2 \in X$, 
        \end{enumerate}
        \item \begin{enumerate}
            \item if $\phi \in X$, then $\neg \phi \in X$,
            \item if $\phi, \psi \in X$ then $(\phi \circ \psi) \in X$ for $\circ \in \{\land,\lor,\to,\leftrightarrow\}$ and
            \item if $x \in \mathcal{V}$ and $\phi \in X$, then $Qx\phi \in X$ for $Q \in \{\forall,\exists\}$.
        \end{enumerate}
    \end{enumerate}
\end{definition}

\begin{definition}[term's primitive term set]\label{def:terms-pt-set}
    Let $\mathcal{S}$ be a first order signature. Then we define the function $tpts_{\mathcal{S}} : \mathcal{T}_{\mathcal{S}} \to \mathcal{P}(\mathcal{P}_{\mathcal{S}})$ mapping terms to the set of primitive terms they contain as
    \begin{enumerate}
        \item $tpts_{\mathcal{S}}(s) = \{s\}$ if $s \in \mathcal{P}_{\mathcal{S}}$ and
        \item $tpts_{\mathcal{S}}(f(t_1,...,t_n)) = tpts_{\mathcal{S}}(t_1)\cup ... \cup tpts_{\mathcal{S}}(t_n)$.
    \end{enumerate}
\end{definition}

\begin{definition}[formula's primitive term set]\label{FO-term-set}
    Let $\mathcal{S}$ be a first order signature. We then define the primitive term set function $pts_{\mathcal{S}} : \mathcal{L}_{\mathcal{S}} \to \mathcal{P}_{\mathcal{S}}$ as 
    \begin{enumerate}
        \item \begin{enumerate}
            \item $pts_{\mathcal{S}}(R(t_1,...,t_n)) = tpts_{\mathcal{S}}(t_1) \cup ... \cup tpts_{\mathcal{S}}(t_n)$,
            \item $pts_{\mathcal{S}}(t_1 = t_2) = tpts_{\mathcal{S}}(t_1) \cup tpts_{\mathcal{S}}(t_2)$,
        \end{enumerate}
        \item \begin{enumerate}
            \item $pts_{\mathcal{S}}(\neg \phi) = pts_{\mathcal{S}}(\phi)$,
            \item $pts_{\mathcal{S}}((\phi \circ \psi)) = pts_{\mathcal{S}}(\phi) \cup pts_{\mathcal{S}}(\psi)$ for $\circ \in \{\land, \lor, \to, \leftrightarrow\}$,
            \item $pts_{\mathcal{S}}(Qx\phi) = pts_{\mathcal{S}}(\phi) - \{x\}$ for $Q \in \{\forall, \exists\}$.
        \end{enumerate}
    \end{enumerate}
\end{definition}

\begin{definition}[substitution function over terms]\label{def:sub-term}
    \uses{def:FO-signature, def:FO-term}
    Let $\mathcal{S} = (\mathcal{C},\mathcal{F},\mathcal{R},ar)$ be a first order signature. Then, we define the substitution function over terms $subt_{\mathcal{S}} : \mathcal{T}_{\mathcal{S}} \times \mathcal{V} \times \mathcal{T}_{\mathcal{S}} \to \mathcal{T}_{\mathcal{S}}$ recursively such that
    \begin{enumerate}
        \item if $s \in \mathcal{P}_{\mathcal{S}}$, then $subt_{\mathcal{S}}(s,x,t) = \begin{cases}
                        s, &\text{if }s \not = x \\
                        t, &\text{if }s = x
                        \end{cases}$
        \item $subt_{\mathcal{S}}(f(t_1,...,t_n),x,t) = f(subt_{\mathcal{S}}(t_1,x,t),...,subt_{\mathcal{S}}(t_n,x,t))$.
    \end{enumerate}
\end{definition}

% TODO: Proceed here with removing explicit references to the logical vocabulary and indexing the functions on signatures.

\begin{definition}[substitution function over formulas]\label{def:sub-f}
    \uses{def:FO-signature, def:FO-term, def:FO-language, def:sub-term}
    Let $\mathcal{S} = (\mathcal{C},\mathcal{F},\mathcal{R},ar)$ a first order signature. Then, we define the substitution function over $\mathcal{L}_{\mathcal{S}}$ as $subf_{\mathcal{S}} : \mathcal{L}_{\mathcal{S}} \times \mathcal{V} \times \mathcal{L}_{\mathcal{S}} \to \mathcal{L}_{\mathcal{S}}$ such that
    \begin{enumerate}
        \item \begin{enumerate} 
            \item $subf_{\mathcal{S}}(R(t_1,...,t_n),x,t) = R(subt_{\mathcal{S}}(t_1,x,t),...,subt_{\mathcal{S}}(t_n,x,t))$,
            \item $subf_{\mathcal{S}}(t_1 = t_2,x,t) = subt_{\mathcal{S}}(t_1,x,t) = subt_{\mathcal{S}}(t_2,x,t)$
        \end{enumerate}
        % TODO : replace sub
        \item \begin{enumerate}
            \item $subf_{\mathcal{S}}(\neg \phi,x,t) = \neg subf_{\mathcal{S}}(\phi,x,t)$,
            \item $subf_{\mathcal{S}}((\phi \circ \psi),x,t) = (subf_{\mathcal{S}}(\phi,x,t) \circ subf_{\mathcal{S}}(\psi,x,t))$, for $\circ \in \{\land,\lor,\to,\leftrightarrow\}$ 
            \item $subf_{\mathcal{S}}(Qy\phi,x,t)= \begin{cases}
                Qy(subf_{\mathcal{S}}(\phi,x,t)) &\text{if } y \not = x \\
                Qy\phi &\text{if } y = x
            \end{cases}$ for $Q \in \{\forall, \exists\}$.
        \end{enumerate}
    \end{enumerate}
\end{definition}

\begin{definition}[inference rule]\label{def:inf-rule}
    \uses{def:formal-language}
    Let $\mathcal{L}$ be a formal language. Then an inference rule defined over $\mathcal{L}$ is a function $r : \mathcal{L}^n \to \mathcal{L}$, where $n \in \mathbb{N}$.
\end{definition}

\begin{definition}[formal system]\label{def:formal-system}
    \uses{def:formal-language, def:inf-rule}
    A formal system $\mathcal{S}$ is a structure $\mathcal{S} = (\mathcal{L}, \mathcal{A}, \mathcal{R})$, where 
    \begin{enumerate}
        \item $\mathcal{L}$ is a formal language,
        \item $\mathcal{A} \subseteq \mathcal{L}$ is a \textit{set of axioms} and 
        \item $\mathcal{R}$ is a \textit{set of inference rules defined over $\mathcal{L}$}.
    \end{enumerate}
\end{definition}

\begin{definition}[sequent set]\label{def:sequent-set}
    \uses{def:formal-language}
    Let $\mathcal{L}$ be a formal language. Then the set of sequents for that language $\mathcal{S} = \mathcal{P}(\mathcal{L}) \times \mathcal{P}(\mathcal{L})$.
\end{definition}

\begin{definition}[sequent calculus inference rules]\label{def:Rsc}
    \uses{def:formal-language, def:sequent-set}
    Let $\mathcal{L}$ be a formal language and $\mathcal{S}$ the sequent set for $\mathcal{L}$. Then the set of inference rules of sequent calculus is the set $\mathcal{R}_{sc}$ containing the functions
    \begin{enumerate}
        \item $r_{ax} : \mathcal{S} \to \mathcal{S}$ defined by $r_{ax}((\emptyset,\emptyset)) = (\{\phi\},\{\phi\})$,
        \item $r_{cut} : \mathcal{S}^2 \to \mathcal{S}$ defined by $r_{cut}((\Gamma,\Delta \cup \{\phi\}),(\{\phi\} \cup \Sigma,\Pi))=(\Gamma \cup \Sigma, \Delta \cup \Pi)$,
        \item $r_{\land L_1} : \mathcal{S} \to \mathcal{S}$ defined by $r_{\land L_1}((\Gamma \cup \{\phi\}, \Delta))=(\Gamma \cup \{\phi \land \psi\}, \Delta)$,
        \item $r_{\land L_2} : \mathcal{S} \to \mathcal{S}$ defined by $r_{\land L_2}((\Gamma \cup \{\psi\}, \Delta))=(\Gamma \cup \{\phi \land \psi\}, \Delta)$,
        \item $r_{\lor L} : \mathcal{S}^2 \to \mathcal{S}$ defined by $r_{\lor L}((\Gamma \cup \{\phi\},\Delta),(\Gamma \cup \{\psi\},\Delta))=(\Gamma \cup \{\phi \lor \psi\},\Delta)$,
        \item $r_{\to L} : \mathcal{S}^2 \to \mathcal{S}$ defined by $r_{\to L}((\Gamma, \{\phi\} \cup \Delta),(\Sigma \cup \{\psi\},\Pi))=(\Gamma \cup \Sigma \cup \{(\phi \to \psi)\},\Delta \cup \Pi)$,
        \item $r_{\neg L} : \mathcal{S} \to \mathcal{S}$ defined by $r_{\neg L}(\Gamma,\{\phi\} \cup \Delta)=(\Gamma \cup \{\neg \phi\},\Delta)$,
        \item $r_{\forall L} : \mathcal{S} \to \mathcal{S}$ defined by $r_{\forall L}(\Gamma \cup \{\phi [t/x]\},\Delta)=(\Gamma \cup \{ \forall x \phi \},\Delta)$,
        \item PROCEED HERE
    \end{enumerate}
\end{definition}

\begin{definition}[natural deduction inference rules]\label{def:Rnd}
    \uses{def:FO-language}
    Let $\mathcal{L}$ be a first order language. Then, the set of inference rules of natural deduction is the set $\mathcal{R}_{nd}$ containing the functions 
    \begin{enumerate}
        \item $r_{\land I} : \mathcal{L}^2 \to \mathcal{L}$ defined by $r_{\land I}(\phi,\psi) = \phi \land \psi$,
        \item $r_{\land E_1} : \mathcal{L} \to \mathcal{L}$ defined by $r_{\land E_1}(\phi \land \psi) = \phi$,
        \item $r_{\land E_2} : \mathcal{L} \to \mathcal{L}$ defined by $r_{\land E_2}(\phi \land \psi) = \psi$,
        \item $r_{\lor I_1} : \mathcal{L} \to \mathcal{L}$ defined by $r_{\lor I_1}(\phi) = \phi \lor \psi$,
        \item $r_{\lor I_2} : \mathcal{L} \to \mathcal{L}$ defined by $r_{\lor I_2}(\psi) = \phi \lor \psi$,
        \item $r_{\lor E} : \mathcal{L}^3 \to \mathcal{L}$ defined by $r_{\lor E}(\phi \lor \psi, \phi \to \omega, \psi \to \omega)=\omega$,
        \item $r_{\to E} : \mathcal{L}^2 \to \mathcal{L}$ defined by $r_{\to E}(\phi \to \psi, \phi)=\psi$,
        \item $r_{\neg I} : \mathcal{L} \to \mathcal{L}$ defined by $r_{\neg I}(\phi \to \bot) = \neg \phi$,
        \item $r_{\neg E} : \mathcal{L} \to \mathcal{L}$ defined by $r_{\neg E}(\phi, \neg \phi)= \bot$,
        \item $r_{\bot E} : \mathcal{L} \to \mathcal{L}$ defined by $r_{\bot E}(\bot)=\phi$ and
        \item $r_{\neg \neg E} : \mathcal{L} \to \mathcal{L}$ defined by $r_{\neg \neg E}(\neg \phi \to \bot)=\phi$,
        \item PROCEED HERE
    \end{enumerate}
\end{definition}

\begin{definition}[\texttt{PA}: the formal system of Peano arithmetic]\label{def:PA}
    \uses{def:LPA, def:formal-system, def:Rnd}
    The formal system of Peano arithmetic \texttt{PA} is defined as $\texttt{PA} = (\mathcal{L}_{PA},\mathcal{A},\mathcal{R}_{nd})$, where $\mathcal{A}$ is the smallest set $X$ such that
    \begin{enumerate}
        \item $\forall x (0 \not = S(x)) \in \mathcal{A}$, 
        \item $\forall x,y (S(x) = S(y) \rightarrow x = y) \in \mathcal{A}$,
        \item $\forall x(x + 0 = x) \in \mathcal{A}$,
        \item $\forall x,y(x + S(y) = S(x+y)) \in \mathcal{A}$,
        \item $\forall x(x \cdot 0 = 0) \in \mathcal{A}$,
        \item $\forall x,y(x \cdot S(y) = x \cdot y + x) \in \mathcal{A}$ and
        \item if $\varphi(x,y_1,...,y_k) \in \mathcal{L}_{PA}$, then $\bar y \Bigl( \bigl(\varphi(0,\bar y) \land \forall x (\varphi(x,\bar y) \to \varphi(S(x),\bar y))\bigr)\Bigr) \to \forall x \varphi(x, \bar y)$, where $\bar y = y_1,...,y_k$
    \end{enumerate}
\end{definition}

\begin{definition}[formal proof]\label{def:formal-proof}
    A formal proof in a logical system $\mathcal{S}$ is a finite sequence of sentences, where each sentence is an axiom of $\mathcal{S}$, an assumption, or follows from the application of one of $\mathcal{S}$'s rules of inference to previous sentences in the sequence (Wikipedia: Formal proof).
\end{definition}

\begin{definition}[provability]
    \label{def:provable-pa}
    \uses{def:formal-proof, def:PA}
    A formula $\varphi$ is provable in a proof system $\mathcal{S}$ if and only if there exists a formal proof $\mathcal{P}$ in $\mathcal{S}$, such that $\mathcal{P}$ contains no assumptions and $\varphi$ is the last sentence of $\mathcal{P}$.
\end{definition}

\begin{definition}[$\mathcal{L}_T$]
    \label{def:LT}
    We define the language $\mathcal{L}_T$ as the language resulting from adding the predicate symbol $T$ to the language $\mathcal{L}$ of \texttt{PA}. 
\end{definition}

\begin{definition}[\texttt{PAT}]
    \label{def:PAT}
    \uses{def:LT}
    We define the system \texttt{PAT} as the system of Peano arithmetic formulated in $\mathcal{L}_T$ including the induction schema for each formula of the language $\mathcal{L}_T$.
\end{definition}
