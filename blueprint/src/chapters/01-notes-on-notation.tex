\chapter{Notes on notation}
Lean is a type-theoretic language and some conceptualizations of common concepts are different in type theory than in set theory. Nevertheless, we want our formalization's documentation to be accessible to mathematicians without a firm background in type theory. Hence, we decided to not make explicit reference to types in the documentation. This is possible because of an intuitive correspondence between types and sets (see e.g. \cite{nederpelt1994}). Hence, whenever the documentation includes a set, the reader can safely assume that set will be implemented in the LEAN code as a type of some sort. 

Moreover, as Lean has additional functionalities on top of its type theoretic basis to allow for efficient programming and proof verification, even more limitations arise preventing the code from following exactly the common conceptualizations of set theoretic mathematics. The most notable of these deviations is that the symbol \lstinline{0} in Lean has a reserved meaning, namely \lstinline{Nat.zero}, an inhabitant of the primitive type \lstinline{Nat}, describing the natural numbers. Note that hence the natural numbers in Lean include 0, the importance of which will become apparent when discussing the implementation of the concept of a language. As we are involved in proving statements about specific axiomatizations of arithmetic we have to define our own notion of the constant 0, and will therefore not use the symbol \lstinline{0} for that but simply the symbol \lstinline{zero}. Note that in this case the \lstinline{zero} in our implementation more closely resembles the symbol 0, because it is just a symbol, and does not carry any extra meaning, which \lstinline{0} does.

Finally we build our formalization on top of work already done by the contributors to the project ``Formalized Formal Logic'' \cite{ffl}. Their implementation of the syntax of first order formulas is based on syntaxes using De Bruijn indices, where quantifiers are not accompanied by the variable they bind, but variables are natural numbers encoding how many levels higher their binding quantifier can be found. For example, the formula 
\begin{align*}
    \forall x (\exists y (P(y) \to P(x)) \land \exists z(R(z) \lor P(x)))
\end{align*}
in regular first order syntax becomes
\begin{align*}
    \forall (\exists (P(0) \to P(1)) \land \exists (R(0) \lor P(1)))
\end{align*}
when making use of De Bruijn indices. As the contributors to \cite{ffl} make use of these De Bruijn indices for first order syntax we do too. 