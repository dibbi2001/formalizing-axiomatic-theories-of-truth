\chapter{Examples}
This chapter contains some example of instances of types defined in the rest of the project.
\section{Prelims}
\subsection{Some terms}
\begin{proposition}\label{ex:13}
    \lean{PA.null}\leanok
    Let $\mathcal{L}_{PA} = \langle F,R \rangle$ be the language of Peano arithmetic. Then $\texttt{zero} \in T_{\mathcal{L}_{PA}}$.
\end{proposition}
\begin{proof}\label{ex:14}
    \leanok
    By Definition \ref{def:lpa}, we have that $\texttt{zero} \in F(0)$. Therefore, by Definition \ref{def:sem-t} clause 3, $\texttt{zero} \in A_{\mathcal{L_{PA}}, \mathbb{N}, 0}$. Hence, by Definition \ref{def:syn-t}, $\texttt{zero} \in T_{\mathcal{L}_{PA}}$. 
\end{proof}
\begin{proposition}
    
\end{proposition}
